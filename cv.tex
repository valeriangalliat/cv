\documentclass{article}

\usepackage[utf8]{inputenc}
\usepackage{etoolbox}
\usepackage{hyperref}

\patchcmd{\subsection}{\bfseries}{\relax}{}{}
\pagenumbering{gobble}

\begin{document}

~

\section*{\huge Valérian Galliat}

~

\begin{itemize}
  \item Montreal, Canada
  \item (+1) 438-868-1762
  \item \href{mailto:val@codejam.info}{val@codejam.info}
  \item \url{https://val.codejam.info/}
\end{itemize}

~

\section*{\en{Work experience}\fr{Expérience professionnelle}}

\subsection*{\textit{Busbud},
             \textbf{\en{Developer}\fr{Développeur}}
             (Montreal, Canada)
             ---
             \small{\en{Since September 2015}\fr{depuis septembre 2015}}}

\en{\begin{itemize}
  \item Maintaining and expanding a fleet of hundreds of adapters for
        external partner APIs, consumed by the main backend API.
  \item Migration of the fleet of adapters from monolith to Docker
        microservices, first on AWS, then on Google Cloud with
        Kubernetes, solving the architecture challenges that comes with it.
  \item Creation and maintenance of internal tools to improve employees
        workflow and efficiency, through CLI tools, internal web apps
        and automated services and bots.
  \item Efficiently tracking and tackling issues at every level of a
        stack composed of dozens of constantly changing moving parts
        with complex state and relationships, whether it's fixing large
        numbers of small hidden bugs or solving bigger architecture
        flaws (e.g. eleminating clear single points of failure by
        redesigning services).
  \item Speccing and architecture design of key parts of a new search
        backend optimized for scalability, performance and high
        cacheability.
  \item Onboarding of new team members, mentoring other colleagues.
\end{itemize}}

\fr{\begin{itemize}
  \item Maintenance et expansion d'une flotte d'une centaine d'adapteurs
         d'API partenaires, utilisés par l'API principale.
  \item Migration des adapteurs d'un système monolythique à une
        infrastructure de microservices Docker, d'abord sur AWS, puis
        sur Google Cloud avec Kubernetes, en gérant les challenges
        architecturels impliqués.
  \item Création et maintenance d'outils internes pour optimiser
        certaines tâches, sous la forme de programmes en ligne de
        commande, applications web internes, services automatisés et
        bots.
  \item Identifier et résoudre efficacement des problèmes à différents
        niveaux de la pile de services, composée de dizaines d'éléments
        en changement constant avec des relations complexes, soit en
        corrigeant une quantité de problèmes isolés, soit en adressant
        de plus gros problèmes architecturels (par exemple éliminer des
        \textit{single point of failure} en refactorant les services).
  \item Spécification et arcitecture de composants clés d'un nouveau
        \textit{backend} de recherche optimisé pour répondre à
        différentes problématiques de performance et caching.
  \item Formation de nouveaux membres de l'équipe, mentoring d'autres
        collègues.
\end{itemize}}

\clearpage

\subsection*{\textit{Dredd},
             \textbf{\en{Developer}\fr{Développeur}}
             (Grenoble, France)
             ---
             \small{\en{September 2012 to July 2015}\fr{de septembre 2012 à juillet 2015}}}

\en{\begin{itemize}
  \item Evolution and maintainance of the \textit{Trackfeeder}
        e-commerce tracking service and its \textit{PrestaShop} extension.
  \item Custom development of efficient import/export scripts for
        several customers, and optimization of existing scripts.
        Documentation and unit tests.
  \item Occasional \textit{Unix} system administration, implementation
        of \textit{Git} in numerous projects, and development of
        deployment scripts.
\end{itemize}}

\fr{\begin{itemize}
  \item Évolution et maintenance de la solution de pilotage e-commerce
        \textit{Trackfeeder} et de son extension \textit{PrestaShop}.
  \item Développements spécifiques de programmes d'import/export de
        données performants pour différents clients et optimisation de
        scripts existants. Documentation et tests unitaires du code.
  \item Administration système \textit{Unix} occasionnelle, mise en
        place de \textit{Git} sur de nombreux projets, développement de
        scripts de déploiement.
\end{itemize}}

~

\section*{\en{Studies}\fr{Études}}

\subsection*{\textit{Aries},
             (Meylan, France)
             ---
             \small{\en{September 2013 to July 2015}\fr{de septembre 2013 à juillet 2015}}}

\begin{itemize}
  \item
    \en{\textbf{Software developer}}\fr{\textbf{Concepteur Développeur Informatique}\footnotemark[1]}
    {\footnotesize (\en{work/study}\fr{alternance})}
\end{itemize}

\subsection*{\textit{Aries},
             (Meylan, France)
             ---
             \small{\en{September 2012 to July 2013}\fr{de septembre 2012 à juillet 2013}}}

\begin{itemize}
  \item
    \textbf{Webmaster}\fr{\footnotemark[2]}
    {\footnotesize (\en{work/study}\fr{alternance})}
\end{itemize}

~

\section*{\en{References}\fr{Références}}

\en{References upon request.}
\fr{Références sur demande.}

\fr{\footnotetext[1]{
  Certification de niveau II (équivalent Bac~+~4) du Ministère chargé de
  l'Emploi.
}

\footnotetext[2]{
  Certification de niveau III (équivalent Bac~+~2) du Ministère chargé de
  l'Emploi.
}}

\clearpage

\section*{\en{Skills}\fr{Compétences}}

\subsection*{\en{Systems}\fr{Systèmes}}

\begin{itemize}
  \item GNU/Linux (Arch Linux, Debian, NixOS)
  \item
    \en{BSD (mostly FreeBSD)}
    \fr{BSD (principalement FreeBSD)}
\end{itemize}

\subsection*{\en{Programming}\fr{Programmation}}

\begin{itemize}
  \item JavaScript (ES6/7, Node.js, TypeScript)
  \item SQL
  \item PHP
  \item Python
  \item Ruby
  \item C
\end{itemize}

\subsection*{Web}

\begin{itemize}
  \item HTML (vanilla, React)
  \item CSS (vanilla, Stylus, SCSS)
\end{itemize}

\subsection*{\en{Databases}\fr{Bases de données}}

\begin{itemize}
  \item PostgreSQL
  \item SQLite
  \item MySQL
\end{itemize}

\subsection*{Infrastructure}

\begin{itemize}
  \item Docker
  \item Google Cloud
  \item Kubernetes
  \item AWS
  \item Fastly (Varnish)
  \item Terraform
\end{itemize}

\subsection*{\en{Softwares}\fr{Logiciels}}

\begin{itemize}
  \item
    \en{Unix ecosystem}
    \fr{Écosystème Unix}
  \item Vim
  \item Git
\end{itemize}

\subsection*{\en{Languages}\fr{Langues}}

\begin{itemize}
  \item
    \en{French (native)}
    \fr{Français (natif)}
  \item
    \en{English}
    \fr{Anglais}
\end{itemize}

\section*{\en{Interests}\fr{Intérêts}}

\begin{itemize}
  \item
    \en{Rock climbing}
    \fr{Escalade}
  \item
    \en{Hiking, trail running}
    \fr{Randonnée}
  \item
    \en{Cycling}
    \fr{Cyclisme}
  \item
    \en{Motorcycle}
    \fr{Moto}
  \item
    \en{Skiing}
    \fr{Ski}
  \item
    \en{Music (guitar, bass, synthesizer)}
    \fr{Musique (guitare, basse, synthétiseur)}
  \item
    \en{Photography}
    \fr{Photographie}
  \item
    \en{Homebrewing}
    \fr{Brassage de bière}
\end{itemize}

\end{document}
